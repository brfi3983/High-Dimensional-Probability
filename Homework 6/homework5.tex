\documentclass[11pt]{article}
\usepackage{amsthm, bookmark, amsmath,textcomp,amssymb,geometry,graphicx,enumerate, mathtools, braket, hyperref, float, listings}
\usepackage[makeroom]{cancel}

\def\Name{Brandon Finley}  % Your name
\def\Homework{5} % Number of Homework
\def\Session{Fall 2020 } % Semester and year
\def\CRS{APPM 5515: High Dimensional Probability}% Course number : course name
\renewcommand\qedsymbol{$\blacksquare$}

\title{\CRS -- \Session --- Homework \Homework} % Course number : course name -- \
\author{\Name}
\markboth{\CRS--\Session\  Homework \Homework\ \Name}{\CRS-- \Session\-- Homework \Homework\ -- \Name}
\pagestyle{myheadings}
\date{}

\textheight=9in
\textwidth=6.5in
\topmargin=-.75in
\oddsidemargin=0.25in
\evensidemargin=0.25in
\setlength\parindent{0pt}
\allowdisplaybreaks

\begin{document}
\maketitle

\begin{enumerate}
	\item Since we know the radius of $B^n(0,1)$ is 1. And that $K(x_o, \varepsilon) = \{ x \in S^{n-1}(1), \langle x,x_0 \rangle \ge \varepsilon \}$, we can use the fact that the radius is one and that $x_1$ is at a height of epsilon. Thus, we can say
	\begin{align*}
		\varepsilon^2 + r^2 = 1 &\implies \left( \frac{x_1}{x_0} \right)^2 + r^2 = 1 \\
		&\implies r^2= 1 - \left( \frac{x_1}{x_0} \right)^2 \\
		&\implies \boxed{r = \sqrt{1 - \left( \frac{x_1}{x_0} \right)^2} = \sqrt{1 - \varepsilon^2}}
	\end{align*}
	\item From lecture, we know the following relationship
	\begin{equation}
		Vol(B^n(r)) = r^n \cdot Vol(B^n(1))
	\end{equation}
	Now, note that our rotationally invariant sphere is also invariant to translation (in terms of measure). Thus, we can simply ignore the center of the sphere and use the above finding.
	Doing this, we can say
	\begin{equation}
		Vol(B^n(x_1, r)) = Vol(B^n(0, r)) = Vol(B^n(r))
	\end{equation}
	Thus,
	\begin{align*}
		Vol(B^n(x_1, r)) &= Vol(B^n(r)) \\
		&= r^n \cdot Vol(B^n(1)) \\
		&= \left[ \sqrt{1 - \left( \frac{x_1}{x_0}\right)^2 } \right]^n \cdot Vol(B^n(1)) \\
		&= \boxed{\left[ \sqrt{1 - \varepsilon^2 }\right]^n \cdot Vol(B^n(1))} \\
	\end{align*}
	\item
	\begin{proof}
	To prove that $C(K) \subset B^n(x_1, r)$, it is equivalent to show that $\Vert x_1 - y_0 \Vert \le r$, where $y_0 \in K(x_0, \varepsilon)$.
	Shown in the figure, since our cap is defined to be dependent on $x_0$, which is dependent on $\varepsilon$, and $x_1 = \varepsilon x_0$, we know that our shifted sphere at $x_1$ will consequently cause our cap and corresponding cone $C(K)$ to be shifted in the same direction (by construction).

	\par Additionally, because it is a sphere, the largest euclidean distance is bounded by the our radius, and any subsequent shift, with a shift at least as large as $\varepsilon$, towards its exterior will lead to a euclidean distance from $x_1$ to its corresponding cap $K(x_0, \epsilon)$ to be less than or equal to $r$ since $\min \langle x, x_0 \rangle = \varepsilon$.
	\par In short, since by construction our $x_1$ is always moving along the axis of $x_0$, the largest euclidean distance from $x_1$ to the spherical cap is its perpendicular axis at the point when $\varepsilon$ is minimized, which in our case is equal to $r$, and so it follows that $\Vert x_0 - y_0 \Vert \le r$.

	\end{proof}
	\item
	\begin{proof}
	Since we know that $C(K) \subset B^n(x_1, r)$, we can say $C(K) \subset B^n(x_1, \sqrt{1 - \varepsilon^2})$.
	And so using question 2, we can say
	\begin{equation*}
	C(K) \subset \left[ \sqrt{1 - \varepsilon^2 }\right]^n B^n(0, 1)
	\end{equation*}
	So taking the volume, we get the inequality
	\begin{align*}
	Vol(C(K)) &\le Vol \left( \left[ \sqrt{1 - \varepsilon^2 }\right]^n B^n(0, 1) \right) \\
	Vol(C(K)) &\le \left[ \sqrt{1 - \varepsilon^2 }\right]^n Vol(B^n(0, 1)) \\
	\end{align*}
	Now, using $1 + x \le e^x$, we know $1 - \varepsilon^2 \le e^{-\varepsilon^2}$. So we say,
	\begin{align*}
		Vol(C(K)) &\le \left[ \sqrt{1 - \varepsilon^2 }\right]^n Vol(B^n(0, 1)) \\
		Vol(C(K)) &\le e^{\frac{-n \varepsilon^2}{2}} Vol(B^n(0, 1)) \\
	\end{align*}
	\end{proof}
	\item
	\begin{proof}
	First, note that we know the following:
	\begin{subequations}
		\begin{equation} \label{eqn:ineq}
			Vol(C(k)) \le e^{\frac{-n \varepsilon^2}{2}} Vol(B^n(0, 1))
		\end{equation}
		\begin{equation}
			\frac{Vol(C(K))}{n} = \sigma(K)
		\end{equation}
		\begin{equation}
			\frac{Vol(B^n(0, 1))}{n}= \sigma(S^{n-1}(1))
		\end{equation}
		\begin{equation}
			\mu^{n-1}(K) = \frac{\sigma(K)}{\sigma(S^{n-1}(1))}
		\end{equation}
	\end{subequations}
	And so, we can use \ref{eqn:ineq} to do the following
	\begin{align*}
	Vol(C(K)) &\le e^{\frac{-n \varepsilon^2}{2}} Vol(B^n(0, 1)) \\
	\sigma(K) \cdot n &\le e^{\frac{-n \varepsilon^2}{2}} \sigma(S^{n-1}(1)) \cdot n \\
	\implies \frac{\sigma(K)}{\sigma(S^{n-1}(1))} &\le e^{\frac{-n \varepsilon^2}{2}} \\
	\implies \mu^{n-1}(K) &\le  e^{\frac{-n \varepsilon^2}{2}} \\
	\implies \mu^{n-1}(K(x_0, \varepsilon)) &\le  e^{\frac{-n \varepsilon^2}{2}} \hspace{0.5cm}\text{(notation)}
	\end{align*}
	\end{proof}
	\item Paul Levy says the optimal bound is
	\begin{equation}
		\le \sqrt{\frac{\pi}{8}} \cdot e^{\frac{-(n - 2)\varepsilon^2}{2}}
	\end{equation}
	And our equation is the following bound
	\begin{equation}
		\le e^{\frac{-n \varepsilon^2}{2}}
	\end{equation}
	Now, note the following
	\begin{align*}
		\sqrt{\frac{\pi}{8}} \cdot e^{\frac{-(n - 2)\varepsilon^2}{2}} = \sqrt{\frac{\pi}{8}} \cdot e^{\frac{-n \varepsilon^2}{2}} \cdot e^{\varepsilon^2} \\
	\end{align*}
	And we know that $ \max \left[ e^{\varepsilon^2} \cdot \sqrt{\frac{\pi}{8}} \right] \approx 0.83 < 1$. Thus,
	\begin{equation}
	\sqrt{\frac{\pi}{8}} \cdot e^{\frac{-(n - 2)\varepsilon^2}{2}} < e^{\frac{-n \varepsilon^2}{2}}
	\end{equation}
	So in summary, our bound is nearly as tight as Levy's bound, with a constant factor as a difference.
\end{enumerate}
\begin{center}
	\boxed{{\textbf{| END |}}}
\end{center}
\end{document}